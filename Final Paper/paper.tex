%% bare_conf.tex
%% V1.4b
%% 2015/08/26
%% by Michael Shell
%% See:
%% http://www.michaelshell.org/
%% for current contact information.
%%
%% This is a skeleton file demonstrating the use of IEEEtran.cls
%% (requires IEEEtran.cls version 1.8b or later) with an IEEE
%% conference paper.
%%
%% Support sites:
%% http://www.michaelshell.org/tex/ieeetran/
%% http://www.ctan.org/pkg/ieeetran
%% and
%% http://www.ieee.org/

%%*************************************************************************
%% Legal Notice:
%% This code is offered as-is without any warranty either expressed or
%% implied; without even the implied warranty of MERCHANTABILITY or
%% FITNESS FOR A PARTICULAR PURPOSE! 
%% User assumes all risk.
%% In no event shall the IEEE or any contributor to this code be liable for
%% any damages or losses, including, but not limited to, incidental,
%% consequential, or any other damages, resulting from the use or misuse
%% of any information contained here.
%%
%% All comments are the opinions of their respective authors and are not
%% necessarily endorsed by the IEEE.
%%
%% This work is distributed under the LaTeX Project Public License (LPPL)
%% ( http://www.latex-project.org/ ) version 1.3, and may be freely used,
%% distributed and modified. A copy of the LPPL, version 1.3, is included
%% in the base LaTeX documentation of all distributions of LaTeX released
%% 2003/12/01 or later.
%% Retain all contribution notices and credits.
%% ** Modified files should be clearly indicated as such, including  **
%% ** renaming them and changing author support contact information. **
%%*************************************************************************


% *** Authors should verify (and, if needed, correct) their LaTeX system  ***
% *** with the testflow diagnostic prior to trusting their LaTeX platform ***
% *** with production work. The IEEE's font choices and paper sizes can   ***
% *** trigger bugs that do not appear when using other class files.       ***                          ***
% The testflow support page is at:
% http://www.michaelshell.org/tex/testflow/



\documentclass[conference]{IEEEtran}
% Some Computer Society conferences also require the compsoc mode option,
% but others use the standard conference format.
%
% If IEEEtran.cls has not been installed into the LaTeX system files,
% manually specify the path to it like:
% \documentclass[conference]{../sty/IEEEtran}





% Some very useful LaTeX packages include:
% (uncomment the ones you want to load)


% *** MISC UTILITY PACKAGES ***
%
%\usepackage{ifpdf}
% Heiko Oberdiek's ifpdf.sty is very useful if you need conditional
% compilation based on whether the output is pdf or dvi.
% usage:
% \ifpdf
%   % pdf code
% \else
%   % dvi code
% \fi
% The latest version of ifpdf.sty can be obtained from:
% http://www.ctan.org/pkg/ifpdf
% Also, note that IEEEtran.cls V1.7 and later provides a builtin
% \ifCLASSINFOpdf conditional that works the same way.
% When switching from latex to pdflatex and vice-versa, the compiler may
% have to be run twice to clear warning/error messages.






% *** CITATION PACKAGES ***
%
%\usepackage{cite}
% cite.sty was written by Donald Arseneau
% V1.6 and later of IEEEtran pre-defines the format of the cite.sty package
% \cite{} output to follow that of the IEEE. Loading the cite package will
% result in citation numbers being automatically sorted and properly
% "compressed/ranged". e.g., [1], [9], [2], [7], [5], [6] without using
% cite.sty will become [1], [2], [5]--[7], [9] using cite.sty. cite.sty's
% \cite will automatically add leading space, if needed. Use cite.sty's
% noadjust option (cite.sty V3.8 and later) if you want to turn this off
% such as if a citation ever needs to be enclosed in parenthesis.
% cite.sty is already installed on most LaTeX systems. Be sure and use
% version 5.0 (2009-03-20) and later if using hyperref.sty.
% The latest version can be obtained at:
% http://www.ctan.org/pkg/cite
% The documentation is contained in the cite.sty file itself.






% *** GRAPHICS RELATED PACKAGES ***
%
\ifCLASSINFOpdf
% \usepackage[pdftex]{graphicx}
% declare the path(s) where your graphic files are
% \graphicspath{{../pdf/}{../jpeg/}}
% and their extensions so you won't have to specify these with
% every instance of \includegraphics
% \DeclareGraphicsExtensions{.pdf,.jpeg,.png}
\else
% or other class option (dvipsone, dvipdf, if not using dvips). graphicx
% will default to the driver specified in the system graphics.cfg if no
% driver is specified.
% \usepackage[dvips]{graphicx}
% declare the path(s) where your graphic files are
% \graphicspath{{../eps/}}
% and their extensions so you won't have to specify these with
% every instance of \includegraphics
% \DeclareGraphicsExtensions{.eps}
\fi
% graphicx was written by David Carlisle and Sebastian Rahtz. It is
% required if you want graphics, photos, etc. graphicx.sty is already
% installed on most LaTeX systems. The latest version and documentation
% can be obtained at: 
% http://www.ctan.org/pkg/graphicx
% Another good source of documentation is "Using Imported Graphics in
% LaTeX2e" by Keith Reckdahl which can be found at:
% http://www.ctan.org/pkg/epslatex
%
% latex, and pdflatex in dvi mode, support graphics in encapsulated
% postscript (.eps) format. pdflatex in pdf mode supports graphics
% in .pdf, .jpeg, .png and .mps (metapost) formats. Users should ensure
% that all non-photo figures use a vector format (.eps, .pdf, .mps) and
% not a bitmapped formats (.jpeg, .png). The IEEE frowns on bitmapped formats
% which can result in "jaggedy"/blurry rendering of lines and letters as
% well as large increases in file sizes.
%
% You can find documentation about the pdfTeX application at:
% http://www.tug.org/applications/pdftex





% *** MATH PACKAGES ***
%
%\usepackage{amsmath}
% A popular package from the American Mathematical Society that provides
% many useful and powerful commands for dealing with mathematics.
%
% Note that the amsmath package sets \interdisplaylinepenalty to 10000
% thus preventing page breaks from occurring within multiline equations. Use:
%\interdisplaylinepenalty=2500
% after loading amsmath to restore such page breaks as IEEEtran.cls normally
% does. amsmath.sty is already installed on most LaTeX systems. The latest
% version and documentation can be obtained at:
% http://www.ctan.org/pkg/amsmath





% *** SPECIALIZED LIST PACKAGES ***
%
%\usepackage{algorithmic}
% algorithmic.sty was written by Peter Williams and Rogerio Brito.
% This package provides an algorithmic environment fo describing algorithms.
% You can use the algorithmic environment in-text or within a figure
% environment to provide for a floating algorithm. Do NOT use the algorithm
% floating environment provided by algorithm.sty (by the same authors) or
% algorithm2e.sty (by Christophe Fiorio) as the IEEE does not use dedicated
% algorithm float types and packages that provide these will not provide
% correct IEEE style captions. The latest version and documentation of
% algorithmic.sty can be obtained at:
% http://www.ctan.org/pkg/algorithms
% Also of interest may be the (relatively newer and more customizable)
% algorithmicx.sty package by Szasz Janos:
% http://www.ctan.org/pkg/algorithmicx




% *** ALIGNMENT PACKAGES ***
%
%\usepackage{array}
% Frank Mittelbach's and David Carlisle's array.sty patches and improves
% the standard LaTeX2e array and tabular environments to provide better
% appearance and additional user controls. As the default LaTeX2e table
% generation code is lacking to the point of almost being broken with
% respect to the quality of the end results, all users are strongly
% advised to use an enhanced (at the very least that provided by array.sty)
% set of table tools. array.sty is already installed on most systems. The
% latest version and documentation can be obtained at:
% http://www.ctan.org/pkg/array


% IEEEtran contains the IEEEeqnarray family of commands that can be used to
% generate multiline equations as well as matrices, tables, etc., of high
% quality.




% *** SUBFIGURE PACKAGES ***
%\ifCLASSOPTIONcompsoc
%  \usepackage[caption=false,font=normalsize,labelfont=sf,textfont=sf]{subfig}
%\else
%  \usepackage[caption=false,font=footnotesize]{subfig}
%\fi
% subfig.sty, written by Steven Douglas Cochran, is the modern replacement
% for subfigure.sty, the latter of which is no longer maintained and is
% incompatible with some LaTeX packages including fixltx2e. However,
% subfig.sty requires and automatically loads Axel Sommerfeldt's caption.sty
% which will override IEEEtran.cls' handling of captions and this will result
% in non-IEEE style figure/table captions. To prevent this problem, be sure
% and invoke subfig.sty's "caption=false" package option (available since
% subfig.sty version 1.3, 2005/06/28) as this is will preserve IEEEtran.cls
% handling of captions.
% Note that the Computer Society format requires a larger sans serif font
% than the serif footnote size font used in traditional IEEE formatting
% and thus the need to invoke different subfig.sty package options depending
% on whether compsoc mode has been enabled.
%
% The latest version and documentation of subfig.sty can be obtained at:
% http://www.ctan.org/pkg/subfig




% *** FLOAT PACKAGES ***
%
%\usepackage{fixltx2e}
% fixltx2e, the successor to the earlier fix2col.sty, was written by
% Frank Mittelbach and David Carlisle. This package corrects a few problems
% in the LaTeX2e kernel, the most notable of which is that in current
% LaTeX2e releases, the ordering of single and double column floats is not
% guaranteed to be preserved. Thus, an unpatched LaTeX2e can allow a
% single column figure to be placed prior to an earlier double column
% figure.
% Be aware that LaTeX2e kernels dated 2015 and later have fixltx2e.sty's
% corrections already built into the system in which case a warning will
% be issued if an attempt is made to load fixltx2e.sty as it is no longer
% needed.
% The latest version and documentation can be found at:
% http://www.ctan.org/pkg/fixltx2e


%\usepackage{stfloats}
% stfloats.sty was written by Sigitas Tolusis. This package gives LaTeX2e
% the ability to do double column floats at the bottom of the page as well
% as the top. (e.g., "\begin{figure*}[!b]" is not normally possible in
% LaTeX2e). It also provides a command:
%\fnbelowfloat
% to enable the placement of footnotes below bottom floats (the standard
% LaTeX2e kernel puts them above bottom floats). This is an invasive package
% which rewrites many portions of the LaTeX2e float routines. It may not work
% with other packages that modify the LaTeX2e float routines. The latest
% version and documentation can be obtained at:
% http://www.ctan.org/pkg/stfloats
% Do not use the stfloats baselinefloat ability as the IEEE does not allow
% \baselineskip to stretch. Authors submitting work to the IEEE should note
% that the IEEE rarely uses double column equations and that authors should try
% to avoid such use. Do not be tempted to use the cuted.sty or midfloat.sty
% packages (also by Sigitas Tolusis) as the IEEE does not format its papers in
% such ways.
% Do not attempt to use stfloats with fixltx2e as they are incompatible.
% Instead, use Morten Hogholm'a dblfloatfix which combines the features
% of both fixltx2e and stfloats:
%
% \usepackage{dblfloatfix}
% The latest version can be found at:
% http://www.ctan.org/pkg/dblfloatfix




% *** PDF, URL AND HYPERLINK PACKAGES ***
%
%\usepackage{url}
% url.sty was written by Donald Arseneau. It provides better support for
% handling and breaking URLs. url.sty is already installed on most LaTeX
% systems. The latest version and documentation can be obtained at:
% http://www.ctan.org/pkg/url
% Basically, \url{my_url_here}.




% *** Do not adjust lengths that control margins, column widths, etc. ***
% *** Do not use packages that alter fonts (such as pslatex).         ***
% There should be no need to do such things with IEEEtran.cls V1.6 and later.
% (Unless specifically asked to do so by the journal or conference you plan
% to submit to, of course. )

\usepackage{hyperref, graphicx}

\usepackage{graphicx}
\usepackage{caption}
%\usepackage{subfig}
\usepackage{subcaption}
\usepackage{listings}
\usepackage{cite}

% correct bad hyphenation here
\hyphenation{op-tical net-works semi-conduc-tor}


\begin{document}
	%
	% paper title
	% Titles are generally capitalized except for words such as a, an, and, as,
	% at, but, by, for, in, nor, of, on, or, the, to and up, which are usually
	% not capitalized unless they are the first or last word of the title.
	% Linebreaks \\ can be used within to get better formatting as desired.
	% Do not put math or special symbols in the title.
	\title{Systems Biology II \\Adapting classical fitting algorithms to cancer-damaged cell oscillations\\}
	
	
	% author names and affiliations
	% use a multiple column layout for up to three different
	% affiliations
	\author{\IEEEauthorblockN{Mathias D. Kallick}
		\IEEEauthorblockA{Department of Computer Science\\
			Colby College\\
			Waterville, ME, 04901\\
			Email: mkallick@colby.edu}
		\and
		\IEEEauthorblockN{Neil K. Sefah}
		\IEEEauthorblockA{Department of Computer Science\\
			Colby College\\
			Waterville, ME, 04901\\
			Email: nksefah@colby.edu}
		\and
		\IEEEauthorblockN{Dakota Thompson}
		\IEEEauthorblockA{Department of Computer Science\\
			Colby College\\
			Waterville, ME, 04901\\
			Email: dthompso@colby.edu}
	}
	
	% conference papers do not typically use \thanks and this command
	% is locked out in conference mode. If really needed, such as for
	% the acknowledgment of grants, issue a \IEEEoverridecommandlockouts
	% after \documentclass
	
	% for over three affiliations, or if they all won't fit within the width
	% of the page, use this alternative format:
	% 
	%\author{\IEEEauthorblockN{Michael Shell\IEEEauthorrefmark{1},
	%Homer Simpson\IEEEauthorrefmark{2},
	%James Kirk\IEEEauthorrefmark{3}, 
	%Montgomery Scott\IEEEauthorrefmark{3} and
	%Eldon Tyrell\IEEEauthorrefmark{4}}
	%\IEEEauthorblockA{\IEEEauthorrefmark{1}School of Electrical and Computer Engineering\\
	%Georgia Institute of Technology,
	%Atlanta, Georgia 30332--0250\\ Email: see http://www.michaelshell.org/contact.html}
	%\IEEEauthorblockA{\IEEEauthorrefmark{2}Twentieth Century Fox, Springfield, USA\\
	%Email: homer@thesimpsons.com}
	%\IEEEauthorblockA{\IEEEauthorrefmark{3}Starfleet Academy, San Francisco, California 96678-2391\\
	%Telephone: (800) 555--1212, Fax: (888) 555--1212}
	%\IEEEauthorblockA{\IEEEauthorrefmark{4}Tyrell Inc., 123 Replicant Street, Los Angeles, California 90210--4321}}
	
	
	
	
	% use for special paper notices
	%\IEEEspecialpapernotice{(Invited Paper)}
	
	
	
	
	% make the title area
	\maketitle
	
	% As a general rule, do not put math, special symbols or citations
	% in the abstract
	\begin{abstract}
		For this project, we were tasked with analyzing oscillations of Per2 and BMal1 in data recorded from cells damaged by cancer - specifically, we wanted to find the period of the oscillations. Michelle Farkas, from the University of Massachusetts Amherst Department of Chemistry, provided us data from cancer-damaged cells. This data was highly irregular and noisy, making a straightforward computational analysis impossible. However, by using a hand-made suite of data cleaning tools and a combination of broad peak searching and sensitive peak detection, we were able to create a piece of software that consistently finds all of the peaks and troughs in the data that was provided to us. This allowed us to produce a function of the period over time for individual datasets.
		
	\end{abstract}
	
	% no keywords
	
	
	
	
	% For peer review papers, you can put extra information on the cover
	% page as needed:
	% \ifCLASSOPTIONpeerreview
	% \begin{center} \bfseries EDICS Category: 3-BBND \end{center}
	% \fi
	%
	% For peerreview papers, this IEEEtran command inserts a page break and
	% creates the second title. It will be ignored for other modes.
	%\IEEEpeerreviewmaketitle

\section{Introduction}
The study of circadian clocks in organisms has been an important endeavor for over 50 years. We examined bio-luminescence recordings from human breast cancer model cells with two circadian reporters: Per2 and Bmal1 (Michelle Farkas, Dept. of Chemistry, University of Massachusetts Amherst). It's important to understand the period of these oscillations as knowing the period plays into understanding the mechanics of the larger cellular systems.  However, due to the nature of the data there were no consistently accurate automated peak detection systems available.  The data has a couple of idiosyncrasies that make automated analysis harder. For example, the oscillations are damped, there was a downwards trend in all of the datasets, and there was a lot of noise in the data. Furthermore, most period finding algorithms are designed to work for data that has a constant period, but we found that this data has a period that varies with time.\\
Moving forward with the task of determining the period we proceeded with several initial steps. This included analyzing smoothing techniques as we needed to smooth the data while ensuring that it remained as true to the actual data as possible through manipulations and derivatives. Additionally, the use of genetic algorithms to create a multiple cosines function for obtaining the periodicity was explored. The classic use of derivatives for finding peaks to calculate the period was also investigated for determining the best automated period detection tool.

\section{Windowing}
\begin{figure}[!ht]
	\centering
	\includegraphics[width=\linewidth]{{"plots/unsmoothed_data"}.png}
	\caption{This figure demonstrates the need for smoothing: we have high frequency noise that confounds analysis, but does not obscure the basic rhythmic pattern.}
	\label{fig:unsmoothed_data}
\end{figure}

One of the core problems we needed to solve before tackling any others was making our data smoother. Initially, all of the data we worked with was noisy and yielded problems when we attempted to detect peaks or find accurate periods. By fixing this, all other computation related to the project would run much more efficiently and with less error. The primary method that we used was smoothing with filters. The first that we got to work properly was the Savitzky-Golay filter, one developed by increasing the signal-to-noise ratio of the plot without entirely corrupting the signal. To do this it essentially generates new fitted subsets of the data according to an equation. While this method has been working pretty effectively for us so far, we wanted to find out whether or not we could find better results by trying to create new filters based on different functions. For this, we turned to window functions; methods that are designed to fit data zero-valued/normalized according to a given interval, using an equation. These were normally used in signal processing as these algorithms are usually geared towards frequency responses and plots for proper spectral analysis (primarily detecting things like spectral leakage and other frequency/sinusoid related plots and problems). Yet, these algorithms could potentially be used to smooth any plot that was given as a signal, depending on its purpose.

However, most of these proved ineffective. The windowing methods we tested were primarily the ones most commonly used in signal processing/general usage. These included the Hamming window function, the Blackman Harris, the Flat-Top, and the Bartlett-Hann functions. 

\begin{figure}[!ht]
	\centering
	\includegraphics[width=\linewidth]{{"plots/grid"}.jpg}
	\caption{Hamming and Flattop Window Test: Using MATLAB code, this shows us one of the sets of data, and shows the effect of using the Hamming and Flattop window functions to dampen it. The bottom result is supposed to show us the periods detected after the application of the algorithm, however, it is not detecting the periods properly, likely due to the cancerous data. This is further confirmed by the period being detected, which is much larger than it should be, as the periods should be 20-30h at best. The explicit use versus the ones matrix indicates if there was an additional matrix used in mitigating the effects of the algorithm on the signal. Explicit use meant no additional matrix was used, ones matrix indicates that ones matrix was used.}
	\label{fig:grid}
\end{figure}

However, all of them yielded problematic results, as they would not give us accurate period detection, no matter what we tried. Considering the nature of these functions (with most of them being derivatives of a core few, which we tested for), we then concluded that these would be ineffective for our purposes, at least in the time-frame we were working in.

We also tried to apply a few methods of dealing with outliers in the data, working with two methods for removing outliers based on standard deviation, and then a custom spike smoothing method which is intended to isolate where data 'spikes' and then modify them according to the adjacent points in the plot, rather than an outright removal. We ended up deciding that smoothing works better for our data in the preliminary stages of our analysis. This result was not entirely unexpected, since the filters we used are the industry standard for smoothing. 


\section{Genetic Algorithms}
One of the most common ways to tackle the period-finding problem is to do curve fitting. That is, find the correct parameters to a function in order to get the difference between that function's outputs and the true data as small as possible. Then, we extract the period from that function (since it should be easy to mathematically determine the period from a set function). Although this method works extremely well, it should also be clear that finding the correct parameters for the function is prohibitively time-consuming (since the potential options are infinite, and testing each set of parameters requires an evaluation of the function at every time step). However, given how common this approach is (and how common the parameter-finding problem is), a large number of algorithms exist to get approximate best parameters in more reasonable computation times.
For this problem, we chose to use a genetic algorithm, which is one of the most common ways to do parameter-finding (also, Mathias had working code from Systems Biology I, so it was relatively easy to implement). Genetic algorithms are meant to emulate the way that evolution works (in the loosest sense of the word). To do this, they take a random sampling of the parameter space, to get the parameters for the first group of parents (each of which is simply a full parameter set for the function, so for example, for Equation \ref{eq:mFourFit}, we would have a C, five As, five {$\phi$}s, and a $\tau$, for each parent). The next step is to evaluate the quality of each parameter set using a cost function (for us, this was simply a least-squares difference between our data and the function defined with each parameter set), and sort the parents according to the quality of their fit. Once the parents have been sorted, we take a subset of the parents according to a selection strategy. What we used for this project was tournament selection - for each parent that we pick, we first pick a random subset of the parents, then we take the best parent from that subset. For this version of a genetic algorithm, we pick the same number of parents as we had in the last round - we just pick the best parents multiple times, and exclude the worst parents, by design. This whole process is then repeated for either a set number of generations (in our case, 10-20), or until the error for the best parameter set becomes lower than some threshold. The best parameter set at the end of this process then becomes the parameters that we use for analysis.

\section{Zielinski et al. - Multiple Cos Fitting}
Given that fitting is the tried and true approach to period-finding, we decided to use it as a starting point for this project. Zielinski, Moore, et al. published a paper detailing various methods of period finding for circadian data \cite{Zielinski}, from which we took our first function, which is called mFourFit. This function (seen in Equation \ref{eq:mFourFit}) makes the assumption that any periodic data can be represented as a linear combination of cosines with different periods. It simply adds together $j$ cosines, where cosine $i$ has period $\tau/i$ and amplitude $A_i$. It also makes the assumption that it is possible to model the data using only periods that are multiples of each other (so there is one base period, $\tau$, and then the rest of the periods are $\tau$ divided by an integer, $i$). An example of a function generated using mFourFit and a random set of parameters can be seen in Figure \ref{fig:mFourFit_ex}. 
Zielinski also described another function, FFT-NLLS, which can be seen in Equation \ref{eq:FFT-NLLS}. FFT-NLLS is very similar to mFourFit, but it doesn't assume that the periods are multiples of each other. Instead, it uses $j$ individual periods that are all found during the parameter-finding phase. This potentially provides a better fit to messier data, but at the same time it dramatically increases the computation cost of parameter-fitting.

\begin{equation} \label{eq:mFourFit}
f(x) = C + \sum_{i=1}^{5} A_i \cos(\frac{2 \pi (t - \phi_i)}{\tau/i})
\end{equation}

\begin{equation} \label{eq:FFT-NLLS}
f(x) = C + \sum_{i=1}^{5} A_i \cos(\frac{2 \pi (t - \phi_i)}{\tau_i})
\end{equation}

\begin{figure}[!ht]
	\centering
	\includegraphics[width=\linewidth]{{"plots/FourFit_GA_match"}.png}
	\caption{A plot of artificially generated data that has then been fit to using the mFourFit equation using a genetic algorithm. Notice that although the fit doesn't perfectly match the artificial data, it does catch every oscillation that it makes (i.e. has a peak everywhere there is a peak and a valley everywhere there is a valley).}
	\label{fig:mFourFit_ex}
\end{figure}

For this project, we used a genetic algorithm to fit parameters for both mFourFit and FFT-NLLS. Unfortunately, it quickly became clear that despite running our genetic algorithms for 10-15 hours, the fitting would simply fit a curve through the middle of our oscillations, picking up the general trend of the data, but not the individual oscillations (that is, totally missing the peaks and valleys). An example of this for FFT-NLLS can be seen in Figure \ref{fig:FFT-NLLS_real}. FourFit looks exactly the same, but trains faster because there are significantly less parameters. This was a truly discouraging result, given that this was the most well-known way of producing the periodicity of data. That said, we were still able to get some interesting information with this fitting, since the fitting would still recognize the overall trend of the data. The downwards trend of the data was a big part of the messiness that made this data hard to work with, so knowing the trend would allow us to correct that trend and work with easier data.

\begin{figure}[!ht]
	\centering
	\includegraphics[width=\linewidth]{{"plots/NLLS_GA_realdata"}.png}
	\caption{A plot of our real data, and the fit we found using a genetic algorithm with FFT-NLLS. The fit does not get the individual oscillations, but it does find the general trend of the data.}
	\label{fig:FFT-NLLS_real}
\end{figure}


\section{Correcting the Data with Polynomial Fit}
When we first realized that we could use the genetic algorithm to correct the data, we were very excited to have solved one of our major issues with the data. However, we soon noticed a major issue - running a fitting algorithm for 10 hours on a single dataset is not the most efficient way to clean data. At that point it is cheaper and easier to have a human do it. Our solution to this problem was to look for a simpler fitting algorithm. After testing a few, we settled on simply using a third-degree polynomial fit to find the trend of the data. This is a much lower level fitting algorithm, and it does not fit complex data nearly as well as the FourFit or FFT-NLLS functions, but our data's downwards trend is pretty simple, so it works very well for that, and, importantly, it fits quickly enough that it is not a noticeable slowdown in our code.
As can be seen in Figure \ref{fig:correcting_with_poly}, the third-degree polynomial fit finds the trend in our data perfectly. If we then invert the fit, and add it to the raw data, we get a corrected dataset (Figure \ref{fig:correcting_with_poly}) that is much, much easier to analyze. This result allowed us to improve our peak finding substantially.

\begin{figure}[!ht]
	\centering
	\includegraphics[width=\linewidth]{{"plots/corrected_poly"}.png}
	\caption{This plot demonstrates the process of using the third-degree polynomial fit to find the downwards trend of our data. We then correct the data based on that downwards trend. We do this by adding the inverse of the trend to our original data.}
	\label{fig:correcting_with_poly}
\end{figure}


\section{Peak Finding with Derivatives}
Standard peak finding techniques have traditionally relied on the first derivative to find peaks and valleys.  This process looks for where the first derivative crosses zero and depending on the direction a peak (positive to negative) or valley (negative to positive) is found.  When applying this technique the first derivative correctly detects most peaks and valleys.  However, due to damping and the downward trend of the data it fails to find peaks that optically appear to be significant in the system's period.  This can be seen in Figure \ref{fig:first_drv_fail} in between the first and last valley peak pairing there appears to be an oscillation in the data that doesn't fully pass the first derivative's threshold.  Thus to find the optically noticeable peaks we needed a more sensitive measure for which we turned to the third derivative.  The third derivative operates in the same manner as the first derivative for finding peaks and valley but it looks at the curvature of the data.  The third derivative correctly detects peaks and valley including the visually significant peaks and valleys.  However, it also detects peaks and valleys where there don't appear to be any significant to the periodicity which can be seen in Figure \ref{fig:third_drv_fail}.  This signified a hybrid method with a sensitivity greater than the pure first derivative and less than the pure third derivative would be required.

\begin{figure}[!ht]
	\centering
	\includegraphics[width=\linewidth]{{"plots/first_drv_fail"}.png}
	\caption{An example of using the first derivative to find peaks and valleys. It is clear that although this finds a good number of peaks and valleys, there are some that were not found. In this case, we notice a bump at about 3 days in, where you can see a valley and then a peak, but the code doesn't catch it.}
	\label{fig:first_drv_fail}
\end{figure}

\begin{figure}[!ht]
	\centering
	\includegraphics[width=\linewidth]{{"plots/third_drv_fail"}.png}
	\caption{An example of using the third derivative to find peaks and valleys. It is clear that although this doesn't miss any peaks or valleys, it is overly sensitive. It not only finds false peaks (around 1.4 days in), but also finds multiple peaks in one spot (for example, at around 2 days in).}
	\label{fig:third_drv_fail}
\end{figure}


\section{Our Peak Detection Process}
The final peak detection method utilized correcting data through polynomial fitting, the first derivative, and the third derivative.  First the data was smoothed using the Savitzky-Golay filter to remove noise from the data.  A third degree polynomial was fit to the dataset, thus identifying the major damping trend.  This third degree polynomial was then inverted and added to the actual data thereby removing the overarching damping and creating a corrected dataset.  By removing the overarching damping trend optical peaks became more susceptible to detection by the first derivative. The peak and valley identification subsequently began by using the first derivative to identify peaks and valleys in the corrected dataset.  The average time between peaks and valleys was then calculated and if a gap between peaks and valleys larger than 1.33 times the average distance was detected it was marked as missing a peaks and/or valleys.  The third derivative was then applied to the uncorrected data strictly within identified gaps to find the remaining peaks and valleys.  The peaks and valleys initially found with the first derivative of the corrected data and those found using the third derivative on the uncorrected data in identified gaps were then consolidated on the uncorrected data to return the set of final peaks and valleys.  With the final set of peaks and valleys the period could then be calculated.  This process is shown in Figure \ref{fig:pv_det_final}.

\begin{figure}[!ht]
	\centering
	\includegraphics[width=\linewidth]{{"plots/Final process"}.png}
	\caption{A diagram of our final peak and valley detection process. This process in total produces a plot of discrete periods, and a function of time that gives us the period. Unlike any of the other methods we tried, this worked equally well for all of our datasets, making it robust enough to be trusted for automation.}
	\label{fig:pv_det_final}
\end{figure}

\section{Using Peaks to Find the Period as a Function of Time}
Having found the peaks and troughs, we need to find the period of the oscillations in the data (since that was our original goal). Normally, the way to turn a set of peaks and troughs into a period would be to find the difference in time between each successive peak and trough and then double that time (since the period is defined as peak-to-peak time, not peak-to-trough time). However, our data does not have a consistent period, so doing that would provide a misleading result. In order to remedy that, we calculated our period for each pair of peaks and troughs (so for peak one to trough one, then from trough one to peak two, and so on). We then plotted this against the time that the peaks were found at (See Figure \ref{fig:per_over_time}). This gives us an idea of the period at discrete points throughout the data, but we wanted to see how the period changed continuously over time as the oscillations occurred. In order to do this, we simply plotted a third-degree fit to the discrete period measurements (again, see Figure \ref{fig:per_over_time}), which allows us to get a sense of the instantaneous change over the data collection window. 

\begin{figure}[!ht]
	\centering
	\includegraphics[width=\linewidth]{{"plots/per_over_time"}.png}
	\caption{A plot of the found period over time of one of our datasets. We also plotted the third-degree polynomial fit of the found period data in order to get a continuous function of the period over time.}
	\label{fig:per_over_time}
\end{figure}


\section{Conclusion}
By using a hand-made suite of data cleaning tools and a combination of broad peak searching and sensitive peak detection, we were able to create a piece of software that consistently finds all of the peaks and troughs in the data that was provided to us. We ended up determining that finding the period in data like this is extremely difficult, particularly because noise makes the derivatives useless, the downward trend of the graph makes some peaks not show up in derivative analysis, and the damping of the oscillations over time makes the last part of the data very hard to analyze. We also noticed that the fact that the period changes over time makes the coding problem harder, because we cannot simply assume that the peaks will be consistently spaced throughout the data, so placing constraints can end up being harder.
However, by smoothing the data using the Savitzky-Golay filter, we removed our issues with noise entirely. We tried using a Genetic Algorithm to fit a combination of multiple cosines to the data, using both the mFourFit (Equation \ref{eq:mFourFit}) and FFT-NLLS (Equation \ref{eq:FFT-NLLS}) equations. However, this simply provided us with a general trend fitting, and not a oscillation fitting. We were able to use polynomial fitting to get the same general trend fitting as the Genetic Algorithms, but with a much, much faster fitting time (10 hours compared to $\textless 1$ second). Using that trend fit, we corrected the downwards slope of the data. We then used first-derivative peak detection to cast a wide net and get the most obvious peaks. By using the mean of the time between peaks, we were able to find places where there should be missing peaks (anywhere the time between found peaks was bigger than $1.3$ times the mean). Inside of those areas, we then used the much more sensitive third-derivative peak detection, which consistently found all of the missing peaks and troughs that first-derivative didn't find. Having a consistent way to find the peaks and troughs in the data allowed us to plot the period over time for all of the data sets that we were given.

% use section* for acknowledgment
\section*{Acknowledgment}


The authors would like to thank Stephanie Taylor for all of her amazing help guiding us through this project. We would also like to thank Michelle Farkas, from the Department of Chemistry at University of Massachusetts Amherst for the data.


% trigger a \newpage just before the given reference
% number - used to balance the columns on the last page
% adjust value as needed - may need to be readjusted if
% the document is modified later
%\IEEEtriggeratref{8}
% The "triggered" command can be changed if desired:
%\IEEEtriggercmd{\enlargethispage{-5in}}

% references section

% can use a bibliography generated by BibTeX as a .bbl file
% BibTeX documentation can be easily obtained at:
% http://mirror.ctan.org/biblio/bibtex/contrib/doc/
% The IEEEtran BibTeX style support page is at:
% http://www.michaelshell.org/tex/ieeetran/bibtex/
%\bibliographystyle{IEEEtran}
% argument is your BibTeX string definitions and bibliography database(s)
%\bibliography{IEEEabrv,../bib/paper}
%
% <OR> manually copy in the resultant .bbl file
% set second argument of \begin to the number of references
% (used to reserve space for the reference number labels box)
\begin{thebibliography}{3}
\bibitem{Zielinski}
Zielinski T, Moore AM, Troup E, Halliday KJ, Millar AJ (2014). \emph{Strengths and Limitations of Period Estimation Methods for Circadian Data}. PLoS ONE 9(5):e96462. doi:10.1371/journal.pone.0096462

\bibitem{SciPy}
Eric Jones and Travis Oliphant and Pearu Peterson and others (2001--). \emph{{SciPy}: Open source scientific tools for {Python}}. "http://www.scipy.org/", {[Online; accessed 5/8/2018]}

\bibitem{Matplotlib}
Hunter, J. D. (2007). \emph{Matplotlib: A 2D graphics environment}. Computing In Science \& Engineering, IEEE COMPUTER SOC, 10.1109/MCSE.2007.55.

\bibitem{MatLab}
(2017). \emph{MATLAB Optimization Toolbox}. The MathWorks, Natick, MA, USA.

\end{thebibliography}

\end{document}